% brice-curriculum_vitae.tex
\synctex=1
\documentclass[11pt]{article}

% preamble.tex
\usepackage[utf8]{inputenc}
\usepackage[T1]{fontenc}
\usepackage[lf]{venturis}

\usepackage{amsmath, amssymb}
\usepackage{fancyhdr}

\usepackage[letterpaper, portrait,
  top=1in, bottom=1in, left=0.96in, right=0.96in
]{geometry}

\renewcommand{\headrulewidth}{0pt}
\renewcommand{\footrulewidth}{1pt}
\fancyhead{}
\lfoot{Daniel Brice (algebraist)}
\cfoot{\doctitle}
\rfoot{Page {\thepage} (of {\pageref{page:last}})}
\pagestyle{fancy}

\newcommand{\setdoctitle}[1]{
  \newcommand{\doctitle}{#1}
}

\newcommand{\makeletterhead}{
	% prints a uniform letterhead for job applications documents
	% #1 is the title of the document (used in the footer)
	\thispagestyle{empty}
  % \setlength{\parskip}{0pt}

	{\LARGE Daniel Brice}\hfill{daniel.brice@gmail.com}

	{(818) 600-2256}
  \hfill
  {\small 5812 Stockdale Hwy Apt 4, Bakersfield, CA 93309}
  \hfill
  {github.com/friedbrice}

  \hrule
}

\renewcommand{\maketitle}{
  \begin{center}
    \doctitle
  \end{center}
}

\newcommand{\sylsec}[2]{%
	% creates a syllabus-style section for course documents
	% #1 is the section title
	% #2 is the section contents

	\vfill
	\begin{minipage}{\textwidth}
	% \setlength{\parskip}{10pt}
	\hspace{-0.2in}\textbf{#1}

	#2
	\end{minipage}
	\vfill
}

\newcommand{\sechead}[1]{%
	% creates just the heading of a section, but not a block
	% #1 is the section title
	% useful for allowing page breaks within section

  \par
  % \vfill
	\hspace{-0.2in}\textbf{#1}
  \par
}

\setlength{\parskip}{2pt}
\setlength{\parindent}{0pt}
\renewcommand{\labelitemi}{\scriptsize${\blacksquare}$}

\newcommand{\syltab}{\hfill{\labelitemi}\hfill}


\begin{document}

\makeletterhead{Curriculum Vitae}

\sylsec{Career Objective}{
  Seeking a faculty position with a Mathematics department.
}

\sylsec{Education}{
  \begin{itemize}
    \item{}
      {\bf Ph.D. Mathematics}
        \syltab Auburn University
        \syltab June 2014.\\
      Advisor: Huajun Huang.\\
      Honors: Baskervil Fellowship, Spring 2009.
    \item{}
      {\bf B.S. Mathematics}
        \syltab California State University, Channel Islands
        \syltab December 2007.\\
      Emphasis: Mathematics Education.\\
      Honors: Mathematics Department Program Honors, Cum Laude.
  \end{itemize}
}

\sylsec{Employment History}{
  \begin{itemize}
    \item{}
      {\bf Assistant Professor of Mathematics}
        \syltab Tuskegee University
        \syltab August 2014 to Present\\
      Responsibilities include:
        12 credit-hours of instruction per semester,
        {\em Pre-Calculus Algebra, Calculus I;}
        advising undergraduates;
        member of Uniform-Final-Exam Committee,
        T-CAEIL (tutoring center) Coordination Committee,
        Textbook-Selection Committee.
    \item{}
      {\bf Teaching Assistant} 
        \syltab Auburn University
        \syltab August 2008 to July 2014\\
      Responsibilities include:
        Instruction of various undergraduate courses,
        {\em Pre-Calculus Algebra, Business Calculus I, Calculus I,
        Calculus II, Calculus III, Math for Elementary Education I;}
        Grading for \emph{Abstract Algebra;}
        Recitation for \emph{Business Calculus I.}
    \item{}
      {\bf Teaching Assistant}
        \syltab California State University, Channel Islands
        \syltab August 2007 to May 2008\\
      Responsibilities include:
        Instruction of {\em College Algebra;}
        Recitation for {\em Abstract Algebra, Real Analysis.}
  \end{itemize}
}

\sylsec{Research Interests}{
  Lie algebras and Lie theory,
  Zero product determined algebras,
  Representation theory.
  Linear and multilinear algebra,
  Tensor decomposition,
  Algebraic logic,
  Category theory,
  Use of games and puzzles
  in building Mathematical reasoning,
  Use of software in providing real-time feedback.
}

\sylsec{Publications}{
  \small
  \begin{itemize}
    \item{} \fullcite{brice2015zero}
    \item{} \fullcite{brice2015derivations}
    \item{} \fullcite{brice2014derivation}
    \item{} \fullcite{brice2011direct}
    \item{} \fullcite{brice0000note}
  \end{itemize}
}

\sylsec{Teaching Experience}{
  8 years experience, 128 credit-hours of undergraduate Mathematics instruction:
  \begin{itemize}
    \item{}
      {\bf Fall 2007}
        {\em College Algebra,}
        Cal State Channel Islands.
    \item{}
      {\bf Spring 2008}
        {\em College Algebra,}
        Cal State Channel Islands.
    \item{}
      {\bf Fall 2008}
        {\em MATHEXCEL} Business Calculus I recitation,
        3 sections, Auburn U.
    \item{}
      {\bf Spring 2009}
        {\em Calculus with Business Applications I,}
        2 sections, Auburn U.
    \item{}
      {\bf Fall 2009}
        {\em Calculus I (Early Transcendentals),}
        2 sections, Auburn U.
    \item{}
      {\bf Spring 2010}
        {\em Calculus II,}
        2 sections, Auburn U.
    \item{}
      {\bf Summer 2010}
        {\em Calculus I (Early Transcendentals),}
        2 sections, Auburn U.
    \item{}
      {\bf Fall 2010}
        {\em Math for Elementary Education I,}
        2 sections, Auburn U.
    \item{}
      {\bf Spring 2011}
        {\em Calculus I (Early Transcendentals),}
        Auburn U.
    \item{}
      {\bf Fall 2011}
        {\em Calculus with Business Applications I,}
        2 sections, Auburn U.
    \item{}
      {\bf Spring 2012}
        {\em Calculus I (Early Transcendentals),}
        2 sections, Auburn U.
    \item{}
      {\bf Summer 2012}
        {\em Pre-Calculus Algebra,}
        2 sections, Auburn U.
    \item{}
      {\bf Fall 2012}
        {\em Calculus III,}
        2 sections, Auburn U.
    \item{}
      {\bf Spring 2013}
        {\em Calculus II,}
        Auburn U.
    \item{}
      {\bf Fall 2013}
        {\em Calculus with Bus App I,}
        Auburn U.
    \item{}
      {\bf Spring 2014}
        {\em Calculus III,}
        2 sections, Auburn U.
    \item{}
      {\bf Summer 2014}
        {\em Calculus I (Early Transcendentals),}
        Auburn U.
    \item{}
      {\bf Fall 2014}\\
        {\em Calculus I (Late Transcendentals),}
        Tuskegee U.\\
        {\em Pre-Calculus Algebra,}
        2 sections, Tuskegee U.
    \item{}
      {\bf Spring 2015}
        {\em Pre-Calculus Algebra,}
        3 sections, Tuskegee U.
  \end{itemize}
}

\sylsec{Additional Mathematics Education Coursework}{
  In addition to the standard coursework of a Mathematics Ph.D., I have completed the
  following courses in Education to develop professionally as a teacher:
  \begin{itemize}
    \item{}
      {\bf Graduate} -- {\em Equity Issues in Mathematics Education; Research in Mathematics Education.}
    \item{}
      {\bf Undergraduate} -- {\em Equity, Diversity \& Foundations of Education; History of Mathematics; Mathematics and Fine Art; Mathematics for Secondary School Teachers.}
  \end{itemize}
}

\sylsec{Community Activities}{
  \begin{itemize}
    \item{}
      {\bf Eagle Scout} --
      {\em 24 April 2002, Troop 110, Rialto CA.}\\
        From Wikipedia: {\em Eagle Scout is the highest rank attainable
        in the Boy Scouting program. Requirements include ...
        demonstrating Scout Spirit through the Boy Scout Oath and Law,
        service, and leadership. This includes an extensive service
        project that the Scout plans, organizes, leads, and manages.}\\
        My service project involved organizing over 30 youth and adult
        volunteers, gathering donations, purchasing materials, and
        coordinating labor for building improvements at Grace Lutheran
        Church in Rialto, CA.
    \item{}
      {\bf Math Club President} --
      {\em Spring 2007, Cal State Channel Islands.}\\
        Scheduled meetings, reserved room, scheduled presenters,
        maintained membership list, maintained website.
    \item{}
      {\bf Auburn Puzzle Party 3} --
      {Winning-team participant}
      {\em Fall 2009, Auburn AL.}\\
        See description below.
    \item{}
      {\bf Auburn Puzzle Party 4} --
      {Volunteer organizer.}
      {\em Fall 2010, Auburn AL.}\\
        See description below.
    \item{}
      {\bf Auburn Puzzle Party 5} --
      {Volunteer organizer.}
      {\em Fall 2012, Auburn AL.}\\
        Auburn supports a thriving community of puzzle-hunters that
        hosts several puzzle-hunts each year. In addition to regular
        participation, I have served as an organizer for two
        puzzle-hunts. I contributed through designing puzzles, event
        production, and event staffing.
    \item{}
      {\bf Linux Club President} --
      {\em Fall 2012, Spring 2013, Auburn U.}\\
        Scheduled meetings, reserved room, scheduled presenters,
        maintained membership list, maintained website.
    \item{}
      {\bf Science Olympiad} --
      {Habitual volunteer.}
      {\em various years, Auburn U.}\\
        Auburn University annually hosts the regional Science Olympiad.
        I have contributed by staffing events.
    \item{}
      {\bf AMP'd Challenge} --
      {Habitual volunteer.}
      {\em various years, Auburn U.}\\
        AMP'd Challenge is an annual mathematics puzzle-hunt for high
        school and middle school student sponsored by the Auburn U.
        College of Sciences and Mathematics. I have contributed by
        designing mathematical puzzles, judging solutions, and staffing
        events.
    \item{}
      {\bf Global Urban Datafest} --
      {Regional winner, global finalist.}
      {\em Spring 2015, Auburn AL.}\\
        Worked on a team with three other amateur programmers to develop
        a data-intensive web application over the course of one weekend.
        We created an app that analyzes webcam images via Canny edge
        detection, gradient vector fields, and principle component
        analysis to detect arbitrary unusual activity. Applications
        include automated surveillance, early-warning systems, and
        disaster recovery. We are currently finalists at the global
        bracket and awaiting the results. Our app is currently live on
        Amazon Web Services. You can view our project page on github:
        {\tt istoomerscornerbeingrolledrightnow.github.io}
  \end{itemize}
}

\sylsec{Technology Skills}{
  \begin{itemize}
    \item{} Experience using Canny edge detection, gradient vector
      field, and principle component analysis to perform advanced
      algorithmic image analysis.
    \item{} Understanding of use of tensor decomposition, singular value
      decomposition, and Perron–Frobenius theorem in structural analysis
      of graphs, particularly as it applies to the study of computer
      networks and the World-wide Web.
    \item{} Understanding of the basic ideas of topological data
      analysis, including persistence homology, and a desire to refine
      this understanding.
    \item{} Linux system administration, BASH shell scripting.
    \item{} Haskell non-strict, purely-functional programming language
      hobbyist/evangelist.
    \item{} Some experience with C and Python and a desire to build
      proficiency.
    \item{} Experience with HTML, CSS, JavaScript languages, Jekyl, Django web frameworks.
    \item{} Experience with Maple, SAGE computer algebra software.
    \item{} Experience with CANVAS, Blackboard course management
      systems.
  \end{itemize}
}

\sylsec{Presentations}{
  \small
  \begin{itemize}
    \item{} \fullcite{pres:brice2015parabolic2}
    \item{} \fullcite{pres:brice2015parabolic1}
    \item{} \fullcite{pres:brice2014derivations}
    \item{} \fullcite{pres:brice2014applications}
    \item{} \fullcite{pres:brice2013constructionsonzero}
    \item{} \fullcite{pres:brice2013constructionson}
    \item{} \fullcite{pres:brice2013zero}
    \item{} \fullcite{pres:brice2013characterizing}
    \item{} \fullcite{pres:brice2013constructions}
    \item{} \fullcite{pres:brice2012derivations}
    \item{} \fullcite{pres:brice2012direct}
    \item{} \fullcite{pres:brice2011symmetry}
    \item{} \fullcite{pres:brice2010continuous}
  \end{itemize}
}

\label{page:last}
\end{document}
