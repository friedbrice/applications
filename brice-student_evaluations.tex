% brice-student_evaluations.tex
\documentclass[11pt]{article}

% preamble.tex
\usepackage[utf8]{inputenc}
\usepackage[T1]{fontenc}
\usepackage[lf]{venturis}

\usepackage{amsmath, amssymb}
\usepackage{fancyhdr}

\usepackage[letterpaper, portrait,
  top=1in, bottom=1in, left=0.96in, right=0.96in
]{geometry}

\renewcommand{\headrulewidth}{0pt}
\renewcommand{\footrulewidth}{1pt}
\fancyhead{}
\lfoot{Daniel Brice (algebraist)}
\cfoot{\doctitle}
\rfoot{Page {\thepage} (of {\pageref{page:last}})}
\pagestyle{fancy}

\newcommand{\setdoctitle}[1]{
  \newcommand{\doctitle}{#1}
}

\newcommand{\makeletterhead}{
	% prints a uniform letterhead for job applications documents
	% #1 is the title of the document (used in the footer)
	\thispagestyle{empty}
  % \setlength{\parskip}{0pt}

	{\LARGE Daniel Brice}\hfill{daniel.brice@gmail.com}

	{(818) 600-2256}
  \hfill
  {\small 5812 Stockdale Hwy Apt 4, Bakersfield, CA 93309}
  \hfill
  {github.com/friedbrice}

  \hrule
}

\renewcommand{\maketitle}{
  \begin{center}
    \doctitle
  \end{center}
}

\newcommand{\sylsec}[2]{%
	% creates a syllabus-style section for course documents
	% #1 is the section title
	% #2 is the section contents

	\vfill
	\begin{minipage}{\textwidth}
	% \setlength{\parskip}{10pt}
	\hspace{-0.2in}\textbf{#1}

	#2
	\end{minipage}
	\vfill
}

\newcommand{\sechead}[1]{%
	% creates just the heading of a section, but not a block
	% #1 is the section title
	% useful for allowing page breaks within section

  \par
  % \vfill
	\hspace{-0.2in}\textbf{#1}
  \par
}

\setlength{\parskip}{2pt}
\setlength{\parindent}{0pt}
\renewcommand{\labelitemi}{\scriptsize${\blacksquare}$}

\newcommand{\syltab}{\hfill{\labelitemi}\hfill}


\begin{document}

\makeletterhead{Selected Student Evaluations}

\begin{center}
\huge Selected Student Evaluations
\end{center}

\vfill

\emph{To the reader: I have attempted to include the evaluations that are most illustrative of my teaching philosophy. Students' typos and spelling are left unchanged. Original documents are available upon request.}

\sechead{Summer 2014 -- Calculus I}
	\begin{itemize}

		\item{} Professor Brice is the first math teacher I think I have actually enjoyed and learned something from. I value his teaching methods and the ways in which he chooses to convey the information.
	\end{itemize}

\sechead{Spring 2014 -- Calculus III}
	\begin{itemize}
		
		\item{} Mr. Brice is a great teacher and he makes students think critically. He explains the big picture where most math teachers just recite the method.
		
		\item{} Mr. Brice is a great instructor. He encourages students to think through the process of the material, not to simply learn and regurgitate. This is reflected in his grading scale, where students cannot barely float along and make a D. Rather unfortunately, this leads to some hurt feelings on the part of students who don't want to work, which is ironic since most students taking this level of math are most likely in majors where critical thinking and problem solving are not only valuable but required. He teaches the course well and encourages students to see him if they need help on the subject matter. Coursework is sometimes confusing or possibly difficult to learn, but, as stated, he assigns homework which can be worked and questions can be asked if needed. Basically, if students want to truly learn and try to do well, seeing him as needed, they should do well in his class.
		
		\item{} Great teacher. Some students may say that he is a difficult test maker. But honestly, his quizzes and tests are some of the most fair ive ever taken at my time at auburn. If you do the homework and understand it all you will do fine on the quizzes. Hes one of the few teachers whos quizzes really match what he assigns on homework, even if it is a slight variation. He does a great job explaining in class and doing step by step examples that are usefully for the quizzes. Student that complain about his calss being overly difficult most likely don't do the homework.
		
		\item{} One of the best Cal teachers in the department hands down. Was very different from what I was used to but I believe that was what worked so well. The fact he encouraged to think critically in his class really prove effective and even helped in other classes. I will encourage any and every student who has yet to take the required class to find Brice and take him.

		\item{} I admire the instructor's method of approach to teaching. The teaching is not the monotonous, arithmetic, computing fashion. The teaching was a more of "well what does this mean" as oppose to "here are the equations, here are the steps, now compute".

Now, initially in the semester the class was shocked by the quiz on the first day, however, now looking back at it, I think it was necessary to see where the class was and that this instructor really wanted to deliver a useful concept to Calculus 3 based on questions.

I enjoyed the moments when the instructor would explain the section and attempt to tie it to engineering. What this does is really motivate the students to understand "what it means", for example, using the tangent plane approximation and using tabular data, or multivariable functions for contour maps; all of which could be used by us. Otherwise, the way I see it, I'm simply learning a new equation and its methods, and then dismissing it after the semester. I, and possibly many others, want to use these concepts as tools for engineering; so I appreciated when the instructor attempted to deliver a concept and applicable approach.

...

Overall, I will walk away with a more conceptual understanding of what the equations and the process through them really means. And because of it, I feel confident to help others that may be struggling with Calculus 3 next semester because I will be able to explain what is going on, as oppose to just displaying the arithmetic steps.
	\end{itemize}


\sechead{Spring 2013 -- Calculus II}
	\begin{itemize}
	
		\item{} Brice is a very good teacher! The class was not necessarily easy but he did a great job really teaching us the concepts of what we were doing instead of just the steps we were supposed to take.
		
		\item{} Very good teacher he does a great job explaining not only how to do things but what exactly we were doing. I would definitely recommend to any student.
	\end{itemize}



\sechead{Fall 2012 -- Calculus III}
	\begin{itemize}
	
		\item{} Mr Brice was fantastic at exploring the material beyond the traditional book computations. He explained many aspects that may be useful later on but not taught in the book.
		
		\item{} Fantastic course. I learned a great deal. Mr. Brice was always approachable and genuinely wanted the class to succeed, not only in the grade dept, but also as analytical thinkers. This was the best class I had all semester.
	
		\item{} I truly enjoyed this class. It wasn't the easiest in the world but I wouldn't expect it to be. Mr Brice however not only cemented the calc 3 material in my mind he finally got me understanding calc 1 and 2 as well and finally got me ready to move on.
		
		\item{} I thought I was going to hate this teacher after he gave us a quiz the first day of class and a test at the end of the first week but it turns out that was just to scare people away that didn't have a firm enough grasp on Cal 2 \emph{[ The student's interpretation of my motives is flawed :-) ]}. This guy turned out to be the best math teacher I've had at Auburn hands down and he even competed with some of my highschool math teachers. Everything about this guy is great. His teaching style, grading style, and personality were all top notch which resulted in me having a stress free semester. I have learned to greatly appreciate good teachers at the collegiate level ... %considering most of them thus far have been embarrassing to their department's and the University as a whole.
		This guy is one of few.
	\end{itemize}



\sechead{Spring 2012 -- Calculus I}
	\begin{itemize}
	
		\item{} Mr. Brice was a great teacher who values education. He was very helpful outside of class as well.
		
		\item{} Professor Brice was an excellent instructor and his teaching methods were extremely helpful. The course material was always explained clearly and concisely.
		
		\item{} Mr. Brice was by far my most helpful teacher this semester. I started out doing poorly in his class and probably would have failed were it not for his dedication to making sure I succeeded. I had to work hard in his class, for he graded strictly and expected a lot out of us, but if I needed help, he was there to provide it. I went to his office hours nearly every time they were available, and if I wanted more help, he would gladly be willing to come in on days where he didn't offer office hours to make sure that I understood the material. During those sessions, he could easily see where I was struggling and was quick to tell me what I needed to change. Due to a combination of his help and my work, I have resurrected my grade in his class. I would recommend him to students who want to become better at calculus. If a student wants an easy grade, don't take him. But if you want to understand math better, he's the one for the job.
	\end{itemize}
	


\sechead{Fall 2011 -- Calculus with Business Applications I}
	\begin{itemize}
	
		\item{} I have really enjoyed this class! As a non-math major taking a class that is more advanced than what is required for my liberal arts major, I was initially worried that this course would prove problematic for me. However, Mr. Brice has made it easy to enjoy calculus with clear explanations of concepts and fair test questions. I am glad that my last academic experience with math was such a pleasurable one!
	\end{itemize}
	
	
\sechead{Spring 2011 -- Calculus I}
	\begin{itemize}
	
		\item{} I transfered from a very small school at attend Auburn, and up until this semester, I thought math was pretty easy. But I realized now that the teachers were making ti that way to pass us. Mr. Brice helped me out whenever I needed it, and tested my thinking process. He is very understanding and makes the work hard so that we will be prepared for the future. he is my favorite teacher this semester, and I \underline{really} appreciate all that he has done for me. I think he cares a great deal about his students, and doesn't want to sugar coat things when students don't need it. I thank him for that honesty and will really miss class. He is a, by far, excellent teacher and I would take him again any day, and recommend him to anyone!
		
		\item{} Can you make this guy teach Cal II this fall. There is no point of having other instructors.
		
		\item{} I am a transfer student here at Auburn University, and I have been through a lot of teachers and hard classes. Out of all of my experiences and classes, I honestly do not believe I have learned as much in one course as I have in Mr Brice's. His class is hard and you must keep up, but it will be worth it in the future.
	\end{itemize}
	
\sechead{Summer 2010 -- Calculus I}
	\begin{itemize}
	
		\item{} The office hours that were held after class helped a lot and made the course easier to understand. If I ever had any questions about anything, he was there to answer them and to help me understand how you get to that answer. He was an excellent teacher and I really enjoyed being a student in his class.
		
		\item{} Very good at teaching the material. Many quizzes instead of tests helped me in understanding the material better. Very easy to understand and good at answering questions.
		
		\item{} Mr. Brice has been my favorite math teacher so far at Auburn. I like the detail in which he teaches each lesson. I also like the idea of having a quiz every few days instead of having a test every two weeks. This encourages me to review the material more often, and as a result get a better grade.
	\end{itemize}
	
\emph{And, finally, one of my all-time favorite evaluations. I understand that this review is not entirely flattering, but it does show my dedication to problems-based learning and the Socratic method.}

	\begin{itemize}
		\item{} As a whole Mr. Brice is a good teacher. His is very helpful and definitely knows the material. Especially one-on-one teaching in  his office hours. he is fantastic. However, he does have flaws: (1) He just seems to drag things out. It always takes a while to get to the real meat of a topic. (2) He does this sometimes. He'll ask us ``If we do this well we get the right answer?'' or ``can we do this?'' and wait forever for a response. It wouldn't be a problem, except that he does it at the beginning of new material. How are we supposed to know how to do something we've never seen before. He's the teacher, not us. It just waists time is all.
		
		Side not: He has awesome hair :)
	\end{itemize}



\label{page:last}
\end{document}
