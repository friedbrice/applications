% brice-curriculum_vitae.tex
\documentclass[11pt]{article}

% preamble.tex
\usepackage[utf8]{inputenc}
\usepackage[T1]{fontenc}
\usepackage{comment}

\usepackage{amsmath, amsthm, amssymb, amsfonts}
\usepackage{graphicx, multicol, xfrac, fancyhdr}

\usepackage[lf]{venturis}

\usepackage[letterpaper, portrait, top=1in, bottom=1.5in, left=0.96in, right=0.96in]{geometry}

\usepackage[backend=bibtex]{biblatex}
\bibliography{references}

\renewcommand{\headrulewidth}{0pt}
\renewcommand{\footrulewidth}{1pt}
\fancyhead{}
\lfoot{Daniel Brice (algebraist, educator)}
\cfoot{\doctitle}
\rfoot{Page {\thepage} (of {\pageref{page:last}})}
\pagestyle{fancy}

\newcommand{\C}{\mathbb{C}}
\newcommand{\R}{\mathbb{R}}
\DeclareMathOperator{\ad}{ad}
\DeclareMathOperator{\Hom}{Hom}

\newcommand{\makeletterhead}[1]{
	% prints a uniform letterhead for job applications documents
	% #1 is the title of the document (used in the footer)
	
	\newcommand{\doctitle}{#1}
	\thispagestyle{empty}
	
	{\huge Daniel Brice} \hfill {\Large Assistant Professor of Mathematics}
	\hrule
	{\small {\tt danielbrice@gmail.com} \hfill (818) 600-2256 \hfill Tuskegee University, Tuskegee Institute AL 36088}
}

\newcommand{\sylsec}[2]{%
	% creates a syllabus-style section for course documents
	% #1 is the section title
	% #2 is the section contents
	
	\vfill
	\begin{minipage}{\textwidth}
	%\setlength{\parskip}{10pt}
	\hspace{-0.2in}\textbf{#1}
	
	#2
	\end{minipage}
	\vfill
}

\newcommand{\sechead}[1]{%
	% creates just the heading of a section, but not a block
	% #1 is the section title
	% useful for allowing page breaks within section
	
	\hspace{-0.2in}\textbf{#1}%
}

\setlength{\parskip}{10pt}
\setlength{\parindent}{0pt}
\renewcommand{\labelitemi}{\scriptsize${\blacksquare}$}

\newcommand{\syltab}{\hfill{\labelitemi}\hfill}


\begin{document}

\setdoctitle{Teaching Experience}

\makeletterhead

\maketitle

Over 8 years of experience teaching college-level Mathematics,
including upper-division courses
and specialized courses for pre-service teachers
and for engineering majors.


\sechead{California State University, Bakersfield}

\begin{itemize}
  \item{}
    Sets and Logic (Winter 2016).\\
    \emph{Proof and Concepts, the fundamentals of abstract mathematics},
    by Dave Witte Morris and Joy Morris.\\
    Investigation of the fundamental tools used in writing mathematical
    proofs, including sentential and predicate calculus, topics from
    naive set theory, Cartesian products, partitions, equivalence
    relations, functions, countability, recursion, the binomial theorem
    and mathematical induction.  This course relies heavily on problem
    solving and writing complete, logically consistent arguments to
    illustrate the correct use of the logical tools and methods
    discussed.

  \item{}
    Calculus II for Engineering Sciences (Winter 2016).\\
    \emph{Calculus: Concepts and Contexts}, by James Stewart.\\
    Introduction to the integral calculus of elementary functions. The
    Fundamental Theorem of Calculus; techniques of integration;
    applications of integration; improper integrals; sequences and
    series. Three dimensional analytic geometry; parametric curves;
    functions of several variables. Applications to Engineering and
    Physics.

  \item{}
    Calculus II (Winter 2016, Fall 2015).\\
    \emph{Calculus: Concepts and Contexts}, by James Stewart.\\
    Introduction to the integral calculus of elementary functions. The
    Fundamental Theorem of Calculus; techniques of integration;
    applications of integration; improper integrals. Three dimensional
    analytic geometry; parametric curves; functions of several variables.
  
  \item{}
    Calculus I for Engineering Sciences (Fall 2015).\\
    \emph{Calculus: Concepts and Contexts}, by James Stewart.\\
    Introduction to the differential calculus of elementary (including
    logarithmic, exponential, and trigonometric) functions. Emphasis on
    limits, continuity, and differentiation. Applications of
    differentiation (including curve sketching, optimization, and
    related rates); antiderivatives. Applications to Engineering and
    Physics.

  \item{}
    Calculus I (Fall 2015).\\
    \emph{Calculus: Concepts and Contexts}, by James Stewart.\\
    Introduction to the differential calculus of elementary (including
    logarithmic, exponential, and trigonometric) functions. Emphasis on
    limits, continuity, and differentiation. Applications of
    differentiation (including curve sketching, optimization, and
    related rates); antiderivatives.

\end{itemize}



\newpage



\sechead{Tuskegee University}

\begin{itemize}
  \item{}
    Analytic Geometry and Calculus I (Fall 2014).\\
    \emph{Calculus}, by James Stewart.\\
    Introduction to analytic geometry; functions; limits; derivatives
    and integrals with some applications.

  \item{}
    College Algebra and Trigonometry I (Fall 2014, Spring 2015).\\
    \emph{Precalculus, A Right Triangle Approach}, by Ratti and McWatters.\\
    Sets; real numbers; absolute value; inequalities; relations and
    functions; polynomial functions, systems of linear equations,
    exponential and logarithmic functions; mathematical induction;
    finite sums and series.

\end{itemize}



\sechead{Auburn University}

\begin{itemize}
  \item{}
    Mathematics for Elementary Education I (Fall 2010).\\
    \emph{Reconceptualizing Mathematics for Elementary Teachers},
    by Sowder and Sowder and Nickerson.\\
    Mathematical insights for elementary school teachers. Sets, the
    structure of the number system (integers, fraction, decimals).

  \item{}
    Calculus III (Spring 2014, Fall 2012).\\
    \emph{Calculus, Early Transcendentals}, by James Stewart.\\
    \emph{University Calculus}, by Hass and Weir.\\
    Multivariate calculus: vector-valued functions, partial derivatives,
    multiple integration, vector calculus.

  \item{}
    Calculus II (Spring 2013, Spring 2010).\\
    \emph{University Calculus}, by Hass and Weir.\\
    Techniques of integration, applications of the integral, parametric
    equations, polar coordinates. Vectors, lines and planes in space.
    Infinite sequences and series.

  \item{}
    Calculus I (Fall 2009, Summer 2010,
    Spring 2011, Spring 2012, Summer 2014).\\
    \emph{Calculus, Early Transcendentals}, by James Stewart.\\
    \emph{University Calculus}, by Hass and Weir.\\
    Limits, the derivative of algebraic, trigonometric, exponential,
    logarithmic functions. Applications of the derivative,
    antiderivatives, the definite integral and applications to area
    problems, the fundamental theorem of calculus.

  \item{}
    Calculus with Business Applications
    (Spring 2009, Fall 2011, Fall 2013).\\
    \emph{Calculus with Business Applications}, by S. T. Tan.\\
    Differentiation and integration of exponential and logarithmic
    functions, applications to business. Functions of several variables,
    partial derivatives, multiple integrals.
  
  \item{}
    Mathexcel Business Calculus Workshop I (Fall 2008).\\
    Workshop for MATH 1680.

  \newpage

  \item{}
    Pre-Calculus Algebra (Summer 2012).\\
    \emph{Precalculus}, by Blitzer.\\
    Preparatory course for calculus. Basic analytic and geometric
    properties of trigonometric functions. Complex numbers, De Moivre's
    Theorem, polar coordinates. No credit is given to students with
    higher-numbered math course.

  \item{}
    Introduction to Abstract Algebra II (Grader, Spring 2013).\\
    Theory of rings and fields, Ideals and Homomorphisms, Quotient
    Rings, Rings of Polynomials, Extensions of Fields, Galois Theory.

  \item{}
    Introduction to Abstract Algebra I (Grader, Fall 2013).\\
    Groups, Groups of Permutations, isomorphisms and homomorphisms;
    Cyclic Groups, Quotient Groups, The Fundamental Homomorphism Theorem.

\end{itemize}



\sechead{California State University, Channel Islands}

\begin{itemize}
  \item{}
    College Algebra (Fall 2007, Spring 2008).\\
    Topic include: basic set theory, number systems and their algebraic
    properties; systems of equations and inequalities; basic analytic
    geometry, matrix algebra and elementary functions; and problem
    solving.

  \item{}
    Abstract Algebra (Teaching Assistant, Spring 2007).\\
    Groups, rings, and fields, the basic algebraic structures in
    contemporary mathematics.

  \item{}
    Real Analysis (Teaching Assistant, Fall 2006).\\
    Topics include: real number system, metric spaces, norms, function
    spaces, continuity, differentiability, integrability of functions,
    sequences and series.

\end{itemize}


\label{page:last}
\end{document}
