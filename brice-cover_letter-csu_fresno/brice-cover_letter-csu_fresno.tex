\documentclass[10pt]{article}

% preamble.tex
\usepackage[utf8]{inputenc}
\usepackage[T1]{fontenc}
\usepackage[lf]{venturis}

\usepackage{amsmath, amssymb}
\usepackage{fancyhdr}

\usepackage[letterpaper, portrait,
  top=1in, bottom=1in, left=0.96in, right=0.96in
]{geometry}

\renewcommand{\headrulewidth}{0pt}
\renewcommand{\footrulewidth}{1pt}
\fancyhead{}
\lfoot{Daniel Brice (algebraist)}
\cfoot{\doctitle}
\rfoot{Page {\thepage} (of {\pageref{page:last}})}
\pagestyle{fancy}

\newcommand{\setdoctitle}[1]{
  \newcommand{\doctitle}{#1}
}

\newcommand{\makeletterhead}{
	% prints a uniform letterhead for job applications documents
	% #1 is the title of the document (used in the footer)
	\thispagestyle{empty}
  % \setlength{\parskip}{0pt}

	{\LARGE Daniel Brice}\hfill{daniel.brice@gmail.com}

	{(818) 600-2256}
  \hfill
  {\small 5812 Stockdale Hwy Apt 4, Bakersfield, CA 93309}
  \hfill
  {github.com/friedbrice}

  \hrule
}

\renewcommand{\maketitle}{
  \begin{center}
    \doctitle
  \end{center}
}

\newcommand{\sylsec}[2]{%
	% creates a syllabus-style section for course documents
	% #1 is the section title
	% #2 is the section contents

	\vfill
	\begin{minipage}{\textwidth}
	% \setlength{\parskip}{10pt}
	\hspace{-0.2in}\textbf{#1}

	#2
	\end{minipage}
	\vfill
}

\newcommand{\sechead}[1]{%
	% creates just the heading of a section, but not a block
	% #1 is the section title
	% useful for allowing page breaks within section

  \par
  % \vfill
	\hspace{-0.2in}\textbf{#1}
  \par
}

\setlength{\parskip}{2pt}
\setlength{\parindent}{0pt}
\renewcommand{\labelitemi}{\scriptsize${\blacksquare}$}

\newcommand{\syltab}{\hfill{\labelitemi}\hfill}


\begin{document}

\makeletterhead

\vfill

\today

\vfill

General Mathematics Search Committee\\
c/o Dr. Maria Nogin\\
Department of Mathematics\\
California State University, Fresno

\vfill

To the members of the General Mathematics Search Committee:

\vfill

I am contacting you today to apply
for the Assistant Professor position
in General Mathematics,
on the advice of Ivona Grzegorczyk,
Chair of Mathematics at Cal State Channel Islands
and Sophia Raczkowski,
Chair of Mathematics at Cal State Bakersfield.
I am currently a lecturer at Bakersfield,
and while I enjoy my teaching responsibilities,
I am seeking a permanent position
that incorporates research
and service responsibilities.

I earned my PhD in Linear Algebra and Lie algebras under the direction
of Huajun Huang at Auburn University in June 2014.
My dissertation research was in structure theory of Lie algebras:
I found a direct sum decomposition of the derivation algebra
of an arbitrary parabolic subalgebra of a reductive Lie algebra,
which I used to show that the derivation algebra
is zero product determined.
Notwithstanding my full-time teaching load as a Lecturer,
I still find some time to maintain a research agenda.
I would very much enjoy the opportunity to further my research efforts
and to work with undergraduates and graduate students
in algebra and related fields
such as computational algebra and data analysis.
As an undergraduate in the Cal State,
research was a formative experience,
essential to my understanding that knowledge
is not merely something to acquire
but rather something to actively and continually strive towards.

I have been teaching university Mathematics since 2007,
including developmental courses and specialized courses for
preservice teachers.
It has been my pleasure to teach at two Cal States
(Channel Islands and Bakersfield) and at the historic
Tuskegee University, and I have used these opportunities to
instill confidence in students from a variety of ethnic and
socio-economic backgrounds.
I attempt to practice a modified form of the flipped classroom,
encouraging my students to read their textbook outside of class,
making time in class for students to engage in collaborative activities
with a nod towards applications, while I make myself available to
offer constructive feedback in the form of questions.
I guess you could call this a problems-based or Socratic approach,
but I think of it as a motivated approach.
If given the time and resources,
I would like to collaborate towards
developing this approach more fully,
and possibly train teaching assistants
and department-level tutors.

I am particularly intrigued by how Fresno State gives the option
of splitting Calculus I into a full-year sequence.
The sheer number of repeats must justifies such an offering,
and I'd be very interested in seeing the numbers on student outcomes
under the two-semester sequence.
I am also impressed by the presence of advanced Mathematics courses
for non-majors, such as \emph{Number Theory for Liberal Studies}
and \emph{Geometry for Liberal Studies.}
I can see myself enjoying teaching either of those,
and developing other similar courses.
At the graduate level, I could teach any
of the algebras, analyses,
\emph{Perspectives in Geometry}
and \emph{Point Set Topology}, among others.

Thank you for the opportunity to apply to Cal Poly Pomona
and for considering my application.
I sincerely hope to join you and the rest
of the Fresno State faculty this August.

\vfill

Respectfully,

\vfill

Daniel Brice\\
\texttt{danielbrice@gmail.com}

\label{page:last}
\end{document}
