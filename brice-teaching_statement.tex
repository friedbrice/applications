% brice-teaching_statement.tex
\documentclass[11pt]{article}

% preamble.tex
\usepackage[utf8]{inputenc}
\usepackage[T1]{fontenc}
\usepackage[lf]{venturis}

\usepackage{amsmath, amssymb}
\usepackage{fancyhdr}

\usepackage[letterpaper, portrait,
  top=1in, bottom=1in, left=0.96in, right=0.96in
]{geometry}

\renewcommand{\headrulewidth}{0pt}
\renewcommand{\footrulewidth}{1pt}
\fancyhead{}
\lfoot{Daniel Brice (algebraist)}
\cfoot{\doctitle}
\rfoot{Page {\thepage} (of {\pageref{page:last}})}
\pagestyle{fancy}

\newcommand{\setdoctitle}[1]{
  \newcommand{\doctitle}{#1}
}

\newcommand{\makeletterhead}{
	% prints a uniform letterhead for job applications documents
	% #1 is the title of the document (used in the footer)
	\thispagestyle{empty}
  % \setlength{\parskip}{0pt}

	{\LARGE Daniel Brice}\hfill{daniel.brice@gmail.com}

	{(818) 600-2256}
  \hfill
  {\small 5812 Stockdale Hwy Apt 4, Bakersfield, CA 93309}
  \hfill
  {github.com/friedbrice}

  \hrule
}

\renewcommand{\maketitle}{
  \begin{center}
    \doctitle
  \end{center}
}

\newcommand{\sylsec}[2]{%
	% creates a syllabus-style section for course documents
	% #1 is the section title
	% #2 is the section contents

	\vfill
	\begin{minipage}{\textwidth}
	% \setlength{\parskip}{10pt}
	\hspace{-0.2in}\textbf{#1}

	#2
	\end{minipage}
	\vfill
}

\newcommand{\sechead}[1]{%
	% creates just the heading of a section, but not a block
	% #1 is the section title
	% useful for allowing page breaks within section

  \par
  % \vfill
	\hspace{-0.2in}\textbf{#1}
  \par
}

\setlength{\parskip}{2pt}
\setlength{\parindent}{0pt}
\renewcommand{\labelitemi}{\scriptsize${\blacksquare}$}

\newcommand{\syltab}{\hfill{\labelitemi}\hfill}


\begin{document}

\makeletterhead{Teaching Statement}

\begin{center}
\huge Teaching Statement
\end{center}

\vfill

In his \emph{Principia Mathematica,} philosopher and mathematician Bertrand Russell argued, contrary to the common wisdom of the time, that the study of Mathematics is not the study of what is necessarily true. Instead, Russel held that Mathematics is the study of what is verbally true based on an understanding of the meanings of the terms used; i.e., that ``mathematical knowledge is, in fact, merely verbal knowledge'' about the meanings and relationships of mathematical terms. Perhaps because of this different foundation, Russell's beliefs about the instruction of Mathematics also diverged from the classical philosophy of the traditional Mathematics classroom, which emphasizes performance of a series of isolated computational problems. Russell believed that conceptual understanding and critical thinking, rather than rote memorization and performance of computational tasks, should be the primary goal of Mathematics instruction.

I feel that Russell's educational philosophy anticipated the world in which we live today, where computational facility seems almost unnecessary to many students---who mistakenly believe that they can find every answer they seek on a greenish-gray screen---but where the ability to translate a problem into mathematical language is what will set them apart from their peers. What good is a calculator to a student that does not know the right question to ask? Let me make myself clear: Computational fluency is an important ingredient for conceptual understanding, but it should not be the sole goal of the Mathematics classroom. So, like Russell, my teaching philosophy centers around not the mere execution of mathematical tasks, but in communicating the meaning underlying each task and how ideas framing those tasks relate to one another.

I believe that students should know what they are doing. I believe that it is valuable for them to understand the underpinnings of what they are doing. To this end, I am particularly influenced by inquiry-based learning, a school of thought which promotes learning by reasoning through questions, constructing thought experiments, and engaging in meaningful dialog. I incorporate these ideas into my classroom whenever possible.

In my classroom, introductions to new topics begin collaboratively, with students working through motivating examples. Unfamiliar terms are introduced using familiar objects, creating relations between concepts with which students are already comfortable. The meaning of a theorem is illustrated in a variety of different situations, and conclusions are attained through discussion. No concept exists in a vacuum: meaning is derived from the context in which an idea is presented. It is my conviction that treating new concepts in the absence of their relationships to familiar concepts obscures the concept at had and encourages a misunderstanding of the purpose of Mathematics.

Throughout the duration of my teaching experience, I have witnessed the damage that the unmotivated approach to teaching Mathematics can inflict, most apparent in every student who considers himself or herself ``not a math person.'' My goal within the classroom is to provide all students, including students who think of themselves in this way, a new opportunity to thrive.

My teaching philosophy emphasizes demystifying Mathematics in the eyes of my students. I want my students to not look to outside authorities, such as a teacher or the back of a textbook, for confirmation, but to their own ability to verify their work themselves; to view themselves as equipped with basic verbal reasoning skills that Mathematics hones; and to think critically and construct their knowledge through exploration of examples and dialog.

Having taught now for over seven years, at institutions as widely varied as the California State University Channel Islands, Auburn University, and Tuskegee University, and having also tutored all levels of Mathematics---grade school to upper-division college---for more than ten years now, has illustrated the efficacy of my choice of pedagogy. My personal experience has demonstrated to me that this approach, placing emphasis on conceptual understanding, allows students who have struggled in the traditional Mathematics classroom to excel in my classroom. I am convinced that the conceptual approach best prepares all students to apply mathematical concepts and reasoning, not only in the Mathematics classroom, but in any academic pursuit, throughout the student's career, and throughout the student's life.

I believe that educators, in general, are in a position to empower their students, and I believe that pedagogy should respect the intelligence of each individual. We are not programming computers, after all. We are communicating ideas with individuals, laying before them the tools with which they may build their knowledge.

\label{page:last}
\end{document}
