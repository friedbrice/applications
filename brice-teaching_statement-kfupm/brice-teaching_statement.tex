% brice-teaching_statement.tex
\documentclass[11pt]{article}

% preamble.tex
\usepackage[utf8]{inputenc}
\usepackage[T1]{fontenc}
\usepackage[lf]{venturis}

\usepackage{amsmath, amssymb}
\usepackage{fancyhdr}

\usepackage[letterpaper, portrait,
  top=1in, bottom=1in, left=0.96in, right=0.96in
]{geometry}

\renewcommand{\headrulewidth}{0pt}
\renewcommand{\footrulewidth}{1pt}
\fancyhead{}
\lfoot{Daniel Brice (algebraist)}
\cfoot{\doctitle}
\rfoot{Page {\thepage} (of {\pageref{page:last}})}
\pagestyle{fancy}

\newcommand{\setdoctitle}[1]{
  \newcommand{\doctitle}{#1}
}

\newcommand{\makeletterhead}{
	% prints a uniform letterhead for job applications documents
	% #1 is the title of the document (used in the footer)
	\thispagestyle{empty}
  % \setlength{\parskip}{0pt}

	{\LARGE Daniel Brice}\hfill{daniel.brice@gmail.com}

	{(818) 600-2256}
  \hfill
  {\small 5812 Stockdale Hwy Apt 4, Bakersfield, CA 93309}
  \hfill
  {github.com/friedbrice}

  \hrule
}

\renewcommand{\maketitle}{
  \begin{center}
    \doctitle
  \end{center}
}

\newcommand{\sylsec}[2]{%
	% creates a syllabus-style section for course documents
	% #1 is the section title
	% #2 is the section contents

	\vfill
	\begin{minipage}{\textwidth}
	% \setlength{\parskip}{10pt}
	\hspace{-0.2in}\textbf{#1}

	#2
	\end{minipage}
	\vfill
}

\newcommand{\sechead}[1]{%
	% creates just the heading of a section, but not a block
	% #1 is the section title
	% useful for allowing page breaks within section

  \par
  % \vfill
	\hspace{-0.2in}\textbf{#1}
  \par
}

\setlength{\parskip}{2pt}
\setlength{\parindent}{0pt}
\renewcommand{\labelitemi}{\scriptsize${\blacksquare}$}

\newcommand{\syltab}{\hfill{\labelitemi}\hfill}


\begin{document}

\setdoctitle{Teaching Statement}

\makeletterhead

\maketitle

\vfill

I prefer teaching methods that empower students. I believe that
education can be used to promote equity and that teaching methods play a
large role in facilitating this goal. Towards this end, I use teaching
methods that encourage a spirit of independence, exploration, and
critical thinking. I believe that all students should come to understand
that, rather than consisting of facts passed down from long ago,
knowledge is something that they can and should actively take a part in
creating. Teaching methods should result in situations where students'
actions and thoughts are relevant. I find that the methods that support
a sense of agency in the student, promoting empowerment for all
students, are those based on inquiry and problem solving.

In his \emph{Principia Mathematica,} philosopher and mathematician
Bertrand Russell argued that Mathematics is the study of what is
verbally true based on an understanding of the meanings of the terms
used; i.e., that ``mathematical knowledge is, in fact, merely verbal
knowledge'' about the meanings and relationships of mathematical terms.
Russell's beliefs about the instruction of Mathematics also diverged
from the classical philosophy of the traditional Mathematics classroom,
which emphasizes performance of a series of isolated computational
problems. Russell believed that conceptual understanding and critical
thinking, rather than rote memorization and performance of computational
tasks, should be the primary goal of Mathematics instruction.

I feel that Russell's educational philosophy anticipated the world in
which we live today, where computational facility seems almost
unnecessary to many students, some of whom mistakenly believe that they
can find every answer they seek on a greenish-gray screen. Others are
left confused, believing that Mathematics consists of seemingly-%
arbitrary rituals as part of a liturgical prompt-response ceremony. Our
teaching methods need to reinforce the idea that Mathematics does not
consist of performing these rote tasks, but of clearly think through
problems and translating them into precise Mathematical language. What
good is a calculator to a student that does not know the right question
to ask? Let me make myself clear: Computational fluency is an important
ingredient for conceptual understanding, but it should not be the sole
or primary goal of the Mathematics classroom. So, like Russell, my
teaching philosophy centers around not the mere execution of
mathematical tasks, but in communicating the meaning underlying each
task and how ideas framing those tasks relate to one another.

For these reasons, I aspire to incorporate the best-practices of
inquiry-based learning in the classroom---learn collaboratively from
examples through progressively deeper questioning. I believe that this
is the best way to empower all students, respecting their individual
agency and promoting equity and diversity. Empower students means giving
them a sense of agency, a sense of control over their learning.
Inquiry-based learning allows students to take control, rather than to
be told.

In my classroom, I assign reading and videos to be completed before
class (drawing from the ideas of the flipped classroom movement).
Introductions to new topics begin with students working collaboratively
through motivating examples. Unfamiliar terms are introduced using
familiar objects, creating relations between concepts with which
students are already comfortable. The meaning of a theorem is
illustrated in a variety of different situations, and conclusions are
attained through discussion. No concept exists in a vacuum: meaning is
derived from the context in which an idea is presented. It is my
conviction that treating new concepts in the absence of their
relationships to familiar concepts obscures the concept at had and
encourages a misunderstanding of the purpose of Mathematics.

Throughout the duration of my teaching experience, I have witnessed the
damage that the unmotivated, procedural approach to teaching Mathematics
can inflict, most apparent in every student who considers himself or
herself ``not a math person.'' My goal within the classroom is to
provide all students, including students who think of themselves in this
way, a new opportunity to thrive. My teaching philosophy emphasizes
demystifying Mathematics in the eyes of my students. I want my students
to not look to outside authorities, such as a teacher or the back of a
textbook, for confirmation, but to their own ability to verify their
work themselves; to view themselves as equipped with basic verbal
reasoning skills that Mathematics hones; and to think critically and
construct their knowledge through exploration of examples and dialog.

It has been my pleasure to have taught across the curriculum to a
diverse body of students at institutions as widely varied as two Cal
States (Channel Islands and Bakersfield), Auburn University, and the
famed HBCU Tuskegee University. Teaching courses from developmental
Mathematics through courses for advanced undergraduates, and specialized
courses for preservice teachers and for engineering students, has
illustrated the efficacy of my choice of pedagogy. My personal
experience confirms to me the body of research on best practices that
shows that this approach, placing emphasis on conceptual understanding,
allows students who have struggled in the traditional Mathematics
classroom to excel in my classroom. I am convinced that the conceptual
approach mediated through inquiry-based learning best prepares all
students to apply mathematical concepts and reasoning, not only in the
Mathematics classroom, but in any academic pursuit, throughout the
student's career, and throughout the student's life.

I believe that educators, in general, are in a position to promote
social equity by empower their students, and I believe that pedagogy
should respect the intelligence of each individual. We are not
programming computers, after all. We are communicating ideas with
individuals, laying before them the tools with which they may build
their knowledge.

\label{page:last}
\end{document}
