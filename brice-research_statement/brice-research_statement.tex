% brice-research_statement.tex
\documentclass[11pt]{article}

% preamble.tex
\usepackage[utf8]{inputenc}
\usepackage[T1]{fontenc}
\usepackage[lf]{venturis}

\usepackage{amsmath, amssymb}
\usepackage{fancyhdr}

\usepackage[letterpaper, portrait,
  top=1in, bottom=1in, left=0.96in, right=0.96in
]{geometry}

\renewcommand{\headrulewidth}{0pt}
\renewcommand{\footrulewidth}{1pt}
\fancyhead{}
\lfoot{Daniel Brice (algebraist)}
\cfoot{\doctitle}
\rfoot{Page {\thepage} (of {\pageref{page:last}})}
\pagestyle{fancy}

\newcommand{\setdoctitle}[1]{
  \newcommand{\doctitle}{#1}
}

\newcommand{\makeletterhead}{
	% prints a uniform letterhead for job applications documents
	% #1 is the title of the document (used in the footer)
	\thispagestyle{empty}
  % \setlength{\parskip}{0pt}

	{\LARGE Daniel Brice}\hfill{daniel.brice@gmail.com}

	{(818) 600-2256}
  \hfill
  {\small 5812 Stockdale Hwy Apt 4, Bakersfield, CA 93309}
  \hfill
  {github.com/friedbrice}

  \hrule
}

\renewcommand{\maketitle}{
  \begin{center}
    \doctitle
  \end{center}
}

\newcommand{\sylsec}[2]{%
	% creates a syllabus-style section for course documents
	% #1 is the section title
	% #2 is the section contents

	\vfill
	\begin{minipage}{\textwidth}
	% \setlength{\parskip}{10pt}
	\hspace{-0.2in}\textbf{#1}

	#2
	\end{minipage}
	\vfill
}

\newcommand{\sechead}[1]{%
	% creates just the heading of a section, but not a block
	% #1 is the section title
	% useful for allowing page breaks within section

  \par
  % \vfill
	\hspace{-0.2in}\textbf{#1}
  \par
}

\setlength{\parskip}{2pt}
\setlength{\parindent}{0pt}
\renewcommand{\labelitemi}{\scriptsize${\blacksquare}$}

\newcommand{\syltab}{\hfill{\labelitemi}\hfill}


\begin{document}

\setdoctitle{Research Statement}

\makeletterhead

\maketitle

\vfill

My dissertation research falls primarily into the categories of linear
algebra, matrix theory, and non-associative algebras. Specifically,
I studied derivations of parabolic Lie algebras and zero product
determined algebras. Prior to writing my dissertation, I co-authored a
paper with my advisor, Dr. Huajun Huang, on zero product determined
algebras \cite{article:brice2015zero}.


\sechead{Zero product determined algebras}

An \emph{algebra} is a vector space $A$ equipped with a bilinear
multiplication operation $\ast$. A bilinear map
$\varphi : A \times A \to V$ (where $V$ is an arbitrary vector space)
is said to have the \emph{zero product property} if
\[
  \forall x, y \in A;\quad
  x \ast y = 0 \text{ implies } \varphi(x,y) = 0
  \text{.}
\]
$A$ is said to be \emph{zero product determined} if for every $\varphi$
that satisfies the zero product property, there is a linear map
$f : A \to V$ such that
\[
  \forall x, y \in A;\quad
  \varphi(x,y) = f(x \ast y)
  \text{.}
\]

Motivated by applications to analysis on Banach algebras, the definition
of zero product determined originated in a 2009 paper of Bre\v{s}ar,
Gra\v{s}i\v{c}, and S\'{a}nches Ortega \cite{brevsar2009zero}.
The authors went on to show that the algebra of $n \times n$ matrices
over a field is zero product determined with respect to the usual matrix
product $XY$, with respect to the Jordan product $X \circ Y = XY + YX$,
and with respect to the Lie product $[X,Y] = XY - YX$.
\footnote{For brevity, I restrict the above discussion to vector spaces
over fields. The authors in \cite{brevsar2009zero} actually prove a more
general result, showing that the matrix algebra over a unital ring $B$
over a commutative unital ring $C$ is (relative to $C$-linearity) zero
product determined with respect to usual matrix multiplication and the
Jordan product, and is ($C$-linear) zero product determined with respect
to the Lie bracket whenever $B$ is ($C$-linear) zero product determined
with respect to the Lie bracket.}
Since the initial work, most research in the field has been in the
direction of classifying zero product determined algebras,
with contributions from Gra\v{s}i\v{c} \cite{gravsivc2010zero}
and from Wang with Yu and Chen \cite{wang2011class}.

My advisor and I contributed, somewhat orthogonally, to this body of
work by examining constructions on algebras that yield zero product
determined algebras \cite{article:brice2015zero}. To achieve this, we cast the
multiplication map $\ast$ as a linear map $\mu : A \otimes A \to A$.
We then proved the following result:
$(A, \mu)$ is zero product determined if and only if the kernel of $\mu$
is generated by rank-one tensors.
Taking advantage of this new setting, we were able to prove that the
direct sum of an arbitrary number of algebras is zero product determined
if and only if each component algebra is zero product determined, that
the tensor product of two zero product determined algebras is zero
product determined,\footnote{Again, we restrict the discussion to
considering vector spaces over fields, but a more general result is in
fact derived.} and various conditions under which homomorphic images of
a zero product determined algebra are zero product determined.
We then defined a class of matrix algebras that generalize block upper
triangular matrices that we call \emph{block upper triangular ladder
matrix algebras}, and we showed that these matrix algebras are zero
product determined with respect to the usual matrix product.

Wang, et. al. in \cite{wang2011class} showed that the parabolic
subalgebras of simple Lie algebras over $\C$ are zero product
determined.
\footnote{Actually, over any algebraically-closed, characteristic-zero
field.}
In my dissertation, I use results of our prior work in
\cite{article:brice2015zero} combined with my results on derivations
(discussed below) to extend the results of Wang, et. al. to parabolic
subalgebras of reductive Lie algebras over $\C$ and to the derivation
algebras of such parabolic subalgebras. I am currently preparing these
results for publication \cite{inprep:brice0000note}.
Beyond that, I hope to determine whether or not the block upper
triangular ladder matrix algebras are zero product determined with
respect to the Jordan product and the Lie bracket.

\sechead{Derivations of parabolic Lie algebras}

A \emph{Lie algebra} consists of a vector space $L$ and a bilinear
multiplication $[\cdot,\cdot]: L \times L \to L$ that satisfies
\begin{itemize}
	\item[] $\forall x \in L;\quad [x,x] = 0$, and
	\item[] $\forall x,y,z \in L;\quad
    \big[x,[y,z]\big]+\big[y,[z,x]\big]+\big[z,[x,y]\big]=0$.
\end{itemize}
A \emph{derivation} on $L$ is a linear map $d: L \to L$ that satisfies
\[
  \forall x,y \in L;\quad
  d\big([x,y]\big) = \big[d(x),y\big] + \big[x,d(y)\big]
  \text{.}
\]

The study of derivations is motivated by connections to cohomology and
the extension problem \cite{farnsteiner1988derivations}.
From the definition of Lie algebra, it may be verified that for each
$x \in L$ the map $y \mapsto [x,y]$ is a derivation on $L$.
We call this map $\ad x$, and we call any such $\ad x$ for $x \in L$ an
\emph{inner derivation}.
All other derivations are termed \emph{outer.}
Characterization of inner derivations is complete as part of the
classical theory of Lie algebras: where $Z(L)$ denotes the center of $L$,
it is well known that $\ad L \cong L / Z(L)$
\cite{humphreys1972introduction}.
Characterization of outer derivations of an arbitrary Lie algebra
remains an open problem.
A significant portion of my dissertation research was to characterize
outer derivations of parabolic subalgebras of reductive Lie algebras
over both $\R$ and $\C$.

Parabolic subalgebras of Lie algebras generalize block upper triangular
matrices, and their study takes a central role in the theory of Lie
algebras due to their connection to representations of $L$
\cite{knapp2002lie}.
In 1972, it was shown by Leger and Luks \cite{leger1972cohomology},
and independently by Tolpygo \cite{tolpygo1972cohomologies},
that all derivations of a parabolic subalgebra of a semisimple Lie
algebra $L$ are inner.
My dissertation extends these results by considering the case where $L$
is reductive rather than semisimple.

Let $P$ be a parabolic subalgebra of a reductive Lie algebra $L$.
Let $\mathcal D$ denote the Lie algebra of derivations on $P$.
We show that $\mathcal D$ decomposes as the direct sum of ideals
\[
  \mathcal D = \mathcal L \oplus \ad P
\]
where $\mathcal L$ consists of all liner maps $d: P \to P$ that map
into $Z(P)$ and map $[P,P]$ to $0$.

Concretely, we explicitly decompose $P$ as a direct sum of subspaces
\[
  P = Z(P) \oplus C \oplus [P,P].
\]
Then $\mathcal L$ takes the form
\[
  \mathcal L \cong \Hom(Z(P) + C,Z(P)).
\]

Using these results, we were able to accomplish our original goal of
showing that $\mathcal D$ is zero product determined;
however, as the problem is interesting in its own right,
I am currently preparing these results for publication
\cite{arxiv:brice2015derivations}.

\sechead{Other potential research directions}

While my dissertation was primarily pure Mathematics,
I believe I could be a valuable asset to a group doing research in
applied Mathematics or in Mathematics education.

Education has always been an interest of mine. In both my undergraduate
and graduate studies, I've taken research-based coursework in
Mathematics eduction. In one course I designed
(though never carried out) a case study dealing with the effects of
calculator use on Pre-Calculus students' grasp of the graphs of the sine
and cosine functions and their transformations. If given the opportunity
to pursue Mathematics education, I would enjoy research into the effects
of (print and software) games on the development of mathematical
reasoning and into software as a means of feedback.

In many ways, linear algebra acts as the interface between pure and
applied Mathematics. Problems in fields as varied as probability,
graph theory, and differential equations are often reduced to questions
of matrix theory, making linear algebra a valuable tool to any research
team. My own interests in applied mathematics would mainly be in the
intersection of linear and multilinear algebra and computer science,
such as the application of multilinear tensor algorithms in web search
or data analysis, or in the use of analog processors in algorithmically
solving computational problems---such as inversion or the extraction of
eigenvalues---in matrix theory

\sechead{References}
\printbibliography[heading=none]

\label{page:last}
\end{document}
