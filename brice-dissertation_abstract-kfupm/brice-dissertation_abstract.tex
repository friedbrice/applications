% brice-dissertation_abstract.tex
\documentclass[11pt]{article}

% preamble.tex
\usepackage[utf8]{inputenc}
\usepackage[T1]{fontenc}
\usepackage{comment}

\usepackage{amsmath, amsthm, amssymb, amsfonts}
\usepackage{graphicx, multicol, xfrac, fancyhdr}

\usepackage[lf]{venturis}

\usepackage[letterpaper, portrait, top=1in, bottom=1.5in, left=0.96in, right=0.96in]{geometry}

\usepackage[backend=bibtex]{biblatex}
\bibliography{references}

\renewcommand{\headrulewidth}{0pt}
\renewcommand{\footrulewidth}{1pt}
\fancyhead{}
\lfoot{Daniel Brice (algebraist, educator)}
\cfoot{\doctitle}
\rfoot{Page {\thepage} (of {\pageref{page:last}})}
\pagestyle{fancy}

\newcommand{\C}{\mathbb{C}}
\newcommand{\R}{\mathbb{R}}
\DeclareMathOperator{\ad}{ad}
\DeclareMathOperator{\Hom}{Hom}

\newcommand{\makeletterhead}[1]{
	% prints a uniform letterhead for job applications documents
	% #1 is the title of the document (used in the footer)
	
	\newcommand{\doctitle}{#1}
	\thispagestyle{empty}
	
	{\huge Daniel Brice} \hfill {\Large Assistant Professor of Mathematics}
	\hrule
	{\small {\tt danielbrice@gmail.com} \hfill (818) 600-2256 \hfill Tuskegee University, Tuskegee Institute AL 36088}
}

\newcommand{\sylsec}[2]{%
	% creates a syllabus-style section for course documents
	% #1 is the section title
	% #2 is the section contents
	
	\vfill
	\begin{minipage}{\textwidth}
	%\setlength{\parskip}{10pt}
	\hspace{-0.2in}\textbf{#1}
	
	#2
	\end{minipage}
	\vfill
}

\newcommand{\sechead}[1]{%
	% creates just the heading of a section, but not a block
	% #1 is the section title
	% useful for allowing page breaks within section
	
	\hspace{-0.2in}\textbf{#1}%
}

\setlength{\parskip}{10pt}
\setlength{\parindent}{0pt}
\renewcommand{\labelitemi}{\scriptsize${\blacksquare}$}

\newcommand{\syltab}{\hfill{\labelitemi}\hfill}


\begin{document}

\setdoctitle{Dissertation Abstract}

\makeletterhead

\maketitle

\vfill

This dissertation builds upon and extends previous work completed by the
author and his advisor in \cite{article:brice2015zero}.
%
A Lie algebra $\mathfrak g$ is said to be zero product determined if for
each bilinear map $\varphi : \mathfrak{g} \times \mathfrak{g} \to V$
that satisfies
\[
  \varphi(x,y)=0 \text{ whenever } [x,y] = 0
\]
there is a linear map $f : [\mathfrak{g}, \mathfrak{g}] \to V$ such that
\[
  \varphi(x,y) = f\big([x,y] \big)
  \text{ for all }
  x, y \in \mathfrak{g}
  \text{.}
\]
%
A derivation $D$ on a Lie algebra $\mathfrak{g}$ is a linear map
$D : \mathfrak{g} \to \mathfrak{g}$ satisfying
\[
  D\big([x,y]\big) = \big[D(x),y\big] + \big[x,D(y)\big]
  \text{ for all }
  x, y \in \mathfrak{g}
  \text{.}
\]
%
$\mathop{\mathrm{Der}} \mathfrak{g}$ denotes the space of all
derivations on the Lie algebra $\mathfrak{g}$,
which itself forms a Lie algebra.

The study of derivations forms part of the classical theory of Lie
algebras and is well understood,
though some work has been done recently that generalizes some of the
classical theory
%
\cite{%
  dixmier1957derivations,
  farnsteiner1988derivations,
  jacobson1955note,
  ou2007derivations,
  sato1965derivations,
  togo1961derivation,
  wang2008derivations,
  wang2006derivations,
  wang2010product,
  zhang2008class%
}.
%
In contrast, the theory of zero product determined algebras is new,
motivated by applications to analysis,
and supports a growing body of literature
%
\cite{%
  alaminos2009maps,
  brevsar2009zero,
  gravsivc2010zero,
  article:brice2015zero,
  wang2011class%
}.
%
In this dissertation, we add to this body of knowledge,
studying the two concepts of derivations and of zero product determined
algebras individually and in relation to each other.

This dissertation contains two main results.
%
Let $K$ denote an algebraically-closed, characteristic-zero field.
%
Let $\mathfrak{q}$ be a parabolic subalgebra of a reductive Lie algebra
$\mathfrak{g}$ over $K$ or $\R$.
%
First we prove a direct sum decomposition of
$\mathop{\mathrm{Der}} \mathfrak{q}$.
%
$\mathop{\mathrm{Der}} \mathfrak{q}$ decomposes as the direct sum of
ideals
\[
  \mathop{\mathrm{Der}} \mathfrak{q}
  = \mathfrak{L} \oplus \ad \mathfrak{q}
  \text{,}
\]
where $\mathfrak{L}$ consists of all linear maps on $\mathfrak{q}$ that
map into the center of $\mathfrak{g}$ and map
$[\mathfrak{q}, \mathfrak{q}]$ to $0$.
%
Second, we apply the decomposition,
along with results of
\cite{article:brice2015zero} and \cite{wang2011class},
to prove that $\mathfrak{q}$ and $\mathop{\mathrm{Der}} \mathfrak{q}$
are zero product determined in the case that $\mathfrak{g}$
is a Lie algebra over $K$.

We conclude by discussing several possible directions for future
research and by applying the main results to providing tabular data for
parabolic subalgebras of reductive Lie algebras of types $A_5$, $G_2$,
and $F_4$.

\sechead{References}
\printbibliography[heading=none]

\label{page:last}
\end{document}
