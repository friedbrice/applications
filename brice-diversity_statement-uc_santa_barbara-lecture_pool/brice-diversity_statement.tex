% brice-teaching_statement.tex
\documentclass[11pt]{article}

% preamble.tex
\usepackage[utf8]{inputenc}
\usepackage[T1]{fontenc}
\usepackage[lf]{venturis}

\usepackage{amsmath, amssymb}
\usepackage{fancyhdr}

\usepackage[letterpaper, portrait,
  top=1in, bottom=1in, left=0.96in, right=0.96in
]{geometry}

\renewcommand{\headrulewidth}{0pt}
\renewcommand{\footrulewidth}{1pt}
\fancyhead{}
\lfoot{Daniel Brice (algebraist)}
\cfoot{\doctitle}
\rfoot{Page {\thepage} (of {\pageref{page:last}})}
\pagestyle{fancy}

\newcommand{\setdoctitle}[1]{
  \newcommand{\doctitle}{#1}
}

\newcommand{\makeletterhead}{
	% prints a uniform letterhead for job applications documents
	% #1 is the title of the document (used in the footer)
	\thispagestyle{empty}
  % \setlength{\parskip}{0pt}

	{\LARGE Daniel Brice}\hfill{daniel.brice@gmail.com}

	{(818) 600-2256}
  \hfill
  {\small 5812 Stockdale Hwy Apt 4, Bakersfield, CA 93309}
  \hfill
  {github.com/friedbrice}

  \hrule
}

\renewcommand{\maketitle}{
  \begin{center}
    \doctitle
  \end{center}
}

\newcommand{\sylsec}[2]{%
	% creates a syllabus-style section for course documents
	% #1 is the section title
	% #2 is the section contents

	\vfill
	\begin{minipage}{\textwidth}
	% \setlength{\parskip}{10pt}
	\hspace{-0.2in}\textbf{#1}

	#2
	\end{minipage}
	\vfill
}

\newcommand{\sechead}[1]{%
	% creates just the heading of a section, but not a block
	% #1 is the section title
	% useful for allowing page breaks within section

  \par
  % \vfill
	\hspace{-0.2in}\textbf{#1}
  \par
}

\setlength{\parskip}{2pt}
\setlength{\parindent}{0pt}
\renewcommand{\labelitemi}{\scriptsize${\blacksquare}$}

\newcommand{\syltab}{\hfill{\labelitemi}\hfill}


\begin{document}

\setdoctitle{Contributions to Diversity Statement}

\makeletterhead

\vspace{0.5in}

\maketitle

\vspace{0.5in}

I believe that education is a social imperative. I believe that
educational opportunities can be used to correct for social inequity.
However, education can be used as a means of protecting the status quo,
favoring a group of privileged individuals over another group of
disadvantaged individuals. Often this occurs unintentionally, through
appeal to traditional methods of teaching or through unconscious bias. I
have contribute to diversity, equity, and inclusion in my own teaching
and conduct by continually seeking to identify my unconscious biases, by
revising my teaching methods in accordance with established best
practices, and by volunteering my time to help design, staff, and
organize community outreach programs for school-aged children. In the
future, I plan on contributing to diversity by mentoring undergraduates
in research and by facilitating their ability to present their research
at conferences and seminars.

One of the principles of our society is the notion that each individual
should have an equal opportunity to develop to their fullest potential,
no matter the circumstances of their birth. Despite this ideal, multiple
achievement gaps, particularly in Mathematics have been identified. A
growing body of evidence suggests that the very lecture format of
teaching is contributing to this gap, as the format unconsciously
assumes and speaks to a particular culture background. In order to close
these achievement gaps, we must adopt teaching methods that facilitate
the learning of all students. Such methods are not exotic, but are
founded on known best-practices. I have prepared to teach a diverse
student body through attending seminars and completing specialized
coursework in Mathematics Education. These suggest that a better format
than the traditional lecture is a Mathematics classroom that combines
guided practice and collaboration towards meaningful goals and maximizes
instructor feedback. I hold high standards for all of my students, and I
ask them to complete complex tasks through scaffolded exploration,
creating opportunities for feedback. In addition to the core course
content, I try to assign meaningful enrichment problems with a view
towards applications, and I encourage collaboration and the use of
technology. I assign and encourage my students to read their textbook
critically outside of class and point them towards useful web resources,
in order to devote class time to activities that allow for meaningful
feedback. In other words, I practice what might be termed as the
beginnings of a flipped classroom. In the words of a peer of mine who
studies Mathematics Education: ``The flipped classroom is nothing new.
You are asking the students to pre-read. Good teaching doesn't change.''
I will add that good teaching creates opportunities for inclusion.

My other contributions to diversity are in outreach and service. As an
undergraduate at Cal State Channel Islands, I worked in collaboration
with a peer to bring problems in graph theory to middle school aged
children in Oxnard, CA. More recently, as a graduate student in Auburn,
AL, I have been a regular volunteer staffer and occasionally an
assistant organizer for the Science Olympiad and the AMP'd Challenge.
These events, particularly the AMP'd Challenge, provide an opportunity
for school-aged children to collaborate on age-appropriate topics from
higher Math, often in the form of puzzles. This helps foster the notion
that an understanding of Mathematics is something that any student can
strive to achieve, regardless of what circumstances that student may
come from. I hope to continue volunteering for and organizing similar
events wherever I make my career, and to continue working closely with
teachers in the community to bring relevant and age-appropriate
contemporary Mathematics into the classroom.

Finally, I would very much like to mentor undergraduates in research,
which will likely involve securing funding from grants for
underrepresented groups. I even look forward to with anticipation my
role in organizing trips to conferences for the students. As an
undergraduate at Cal State Channel Islands, my experiences with
undergraduate research and attending and presenting at conferences where
a formative part of my education. The tireless efforts of my professors
created opportunities for me: I was able to visit places and meet people
that I previously believed I could not. I wish to create the same
opportunities for my students in my future career.

\label{page:last}
\end{document}
