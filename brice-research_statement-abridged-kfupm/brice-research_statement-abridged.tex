% brice-research_statement.tex
\documentclass[11pt]{article}

% preamble.tex
\usepackage[utf8]{inputenc}
\usepackage[T1]{fontenc}
\usepackage{comment}

\usepackage{amsmath, amsthm, amssymb, amsfonts}
\usepackage{graphicx, multicol, xfrac, fancyhdr}

\usepackage[lf]{venturis}

\usepackage[letterpaper, portrait, top=1in, bottom=1.5in, left=0.96in, right=0.96in]{geometry}

\usepackage[backend=bibtex]{biblatex}
\bibliography{references}

\renewcommand{\headrulewidth}{0pt}
\renewcommand{\footrulewidth}{1pt}
\fancyhead{}
\lfoot{Daniel Brice (algebraist, educator)}
\cfoot{\doctitle}
\rfoot{Page {\thepage} (of {\pageref{page:last}})}
\pagestyle{fancy}

\newcommand{\C}{\mathbb{C}}
\newcommand{\R}{\mathbb{R}}
\DeclareMathOperator{\ad}{ad}
\DeclareMathOperator{\Hom}{Hom}

\newcommand{\makeletterhead}[1]{
	% prints a uniform letterhead for job applications documents
	% #1 is the title of the document (used in the footer)
	
	\newcommand{\doctitle}{#1}
	\thispagestyle{empty}
	
	{\huge Daniel Brice} \hfill {\Large Assistant Professor of Mathematics}
	\hrule
	{\small {\tt danielbrice@gmail.com} \hfill (818) 600-2256 \hfill Tuskegee University, Tuskegee Institute AL 36088}
}

\newcommand{\sylsec}[2]{%
	% creates a syllabus-style section for course documents
	% #1 is the section title
	% #2 is the section contents
	
	\vfill
	\begin{minipage}{\textwidth}
	%\setlength{\parskip}{10pt}
	\hspace{-0.2in}\textbf{#1}
	
	#2
	\end{minipage}
	\vfill
}

\newcommand{\sechead}[1]{%
	% creates just the heading of a section, but not a block
	% #1 is the section title
	% useful for allowing page breaks within section
	
	\hspace{-0.2in}\textbf{#1}%
}

\setlength{\parskip}{10pt}
\setlength{\parindent}{0pt}
\renewcommand{\labelitemi}{\scriptsize${\blacksquare}$}

\newcommand{\syltab}{\hfill{\labelitemi}\hfill}


\begin{document}

\setdoctitle{Research Statement}

\makeletterhead

\maketitle

\vfill

My research interests lie primarily with non-associative algebras---%
mostly structure theory of Lie algebras, but also algebras more
generally. My current pursuits align along two axes: zero product
determined algebras and derivations of parabolic Lie algebras.

\sechead{Zero product determined algebras}

For our purposes, an \emph{algebra} is a vector space $A$ equipped with
a bilinear (not-necessarily-associative) multiplication operation
$\ast$.
A bilinear map $\varphi : A \times A \to V$
(where $V$ is an arbitrary vector space)
is said to have the \emph{zero product property} if
\[
  \forall x, y \in A;\quad
  x \ast y = 0 \text{ implies } \varphi(x,y) = 0
  \text{.}
\]
$A$ is said to be \emph{zero product determined} if for every $\varphi$
that satisfies the zero product property, there is a linear map
$f : A \to V$ such that
\[
  \forall x, y \in A;\quad
  \varphi(x,y) = f(x \ast y)
  \text{.}
\]

Motivated by the linear preserver problem of operator theory, the
concept of a zero product determined algebra originated in a 2009 paper
of Bre\v{s}ar, Gra\v{s}i\v{c}, and S\'{a}nches Ortega
\cite{brevsar2009zero}. The authors went on to show that the algebra of
$n \times n$ matrices over a field is zero product determined with
respect to the usual matrix product $XY$, with respect to the Jordan
product $X \circ Y = XY + YX$, and with respect to the Lie bracket
$[X,Y] = XY - YX$.

Huajun Huang and I contributed by examining constructions on algebras
that yield zero product determined algebras
\cite{article:brice2015zero}. We showed that an algebra $(A, \ast)$ is
zero product determined if and only if the kernel of the map $\mu : x
\otimes y \mapsto x \ast y$ has a basis consisting of rank-one tensors.
Using this criterion, we showed that the direct sum of an arbitrary
number of algebras is zero product determined if and only if each
summand is zero product determined and that the tensor product of
two zero product determined algebras is zero product determined, and we
identified various conditions under which homomorphic images of a zero
product determined algebra are zero product determined. We then defined
a class of matrix algebras that generalize block upper triangular
matrices that we call \emph{block upper triangular ladder matrix
algebras}, and we showed that these matrix algebras are zero product
determined with respect to the usual matrix product.

Wang, et. al. in \cite{wang2011class} showed that the parabolic
subalgebras of simple Lie algebras over $\C$ are zero product
determined. I use results of  \cite{article:brice2015zero} combined with
my results on derivations \cite{arxiv:brice2015derivations} to extend
the results of Wang, et. al. to parabolic subalgebras of reductive Lie
algebras over $\C$ and to the derivation algebras of such parabolic
subalgebras. I am currently preparing these results for publication
\cite{inprep:brice0000note}.

I am currently investigating whether the block upper triangular ladder
matrix algebras are zero product determined under the Lie bracket, and I
would like to consider the same under the Jordan product as well. More
generally, I am investigating conditions under which the semi-direct sum
of two algebras is zero product determined. I would also like to further
develop the relationship between zero product determined algebras and
the linear preserver problem, perhaps jointly with experts in operator
theory.

\sechead{Derivations of parabolic Lie algebras}

A \emph{Lie algebra} consists of a vector space $\mathfrak{g}$
and a bilinear multiplication
$[\cdot,\cdot]: \mathfrak{g} \times \mathfrak{g} \to \mathfrak{g}$
that satisfies
\begin{itemize}
	\item[] $\forall x \in \mathfrak{g};\quad [x,x] = 0$, and
	\item[] $\forall x,y,z \in \mathfrak{g};\quad
    \big[x,[y,z]\big]+\big[y,[z,x]\big]+\big[z,[x,y]\big]=0$.
\end{itemize}
A \emph{derivation} on $\mathfrak{g}$ is a linear map
$d: \mathfrak{g} \to \mathfrak{g}$ that satisfies
\[
  \forall x,y \in \mathfrak{g};\quad
  d\big([x,y]\big) = \big[d(x),y\big] + \big[x,d(y)\big]
  \text{.}
\]

For each $x \in \mathfrak{g}$ the map $\ad x : y \mapsto [x,y]$ is a
derivation on $L$. Any such $\ad x$ for $x \in \mathfrak{g}$ is termed
an \emph{inner derivation}, and all other derivations are termed
\emph{outer.} Characterization of outer derivations of an arbitrary Lie
algebra remains an open problem. A significant portion of my
dissertation research was to characterize outer derivations of parabolic
subalgebras of $\mathfrak{g}$ when $\mathfrak{g}$ is a reductive Lie
algebra over $\R$ and $\C$.

Parabolic subalgebras of Lie algebras generalize block upper triangular
matrices, and their study takes a central role in the theory of Lie
algebras. In 1972, it was known that all derivations of a parabolic
subalgebra of a semisimple Lie algebra $\mathfrak{g}$ are inner
\cite{leger1972cohomology}. My dissertation extends these results by
considering the case where $\mathfrak{g}$ is reductive.

I showed that an arbitrary derivation of a parabolic $\mathfrak q$ is
the sum of an inner derivation and a linear map sending $\mathfrak q$
into $Z(\mathfrak q)$ and sending $[\mathfrak q, \mathfrak q]$ to $0$.
Conversely, any such linear map is an outer derivation, and the
derivation algebra $\mathcal{D}(\mathfrak{q})$ decomposes as the Lie
algebra direct sum $\mathcal{D}(\mathfrak{q}) \cong \ad q \oplus
\mathcal{L}$ where $\mathcal{L}$ consists of all linear maps from
$\mathfrak{q}$ to $Z(\mathfrak{q})$ that kill
$[\mathfrak{q},\mathfrak{q}]$. These results were submitted to the
\emph{Journal of Lie Theory} in November 2015
\cite{arxiv:brice2015derivations}. I intend to extend these results to
reductive Lie algebras over fields of finite characteristic.

\sechead{References}
\printbibliography[heading=none]

\label{page:last}
\end{document}
